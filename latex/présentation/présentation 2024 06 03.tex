\documentclass{beamer}

\usepackage[english]{babel}
\usepackage[utf8x]{inputenc}
\usepackage{hyperref, % clickable links
    graphicx, % include images
    listings, % for code and formatting
    caption, % customization of captions in figures and tables
    stackengine, % custom layouts 
    amsmath, % math env
    xcolor, % extend color support
    multicol, % multiple columns layout
    booktabs, % high quality tables
    lipsum % remove it
}
\usepackage{listings}
\usepackage[listings,skins]{tcolorbox}

\usepackage{Ensae_beamer} % customized style


\usepackage{graphicx}
\usepackage{algorithm}
\usepackage{algpseudocode}
\usepackage{caption}
\usepackage{subcaption}
\usepackage{pifont}
    \usetikzlibrary{positioning}
    \usetikzlibrary{backgrounds}
    \usetikzlibrary{arrows.meta}
\usetikzlibrary{3d, calc}


\usepackage{graphicx}
\usepackage[textwidth=8em,textsize=small]{todonotes}
\usepackage{amsmath}
\usepackage{amssymb}
\usepackage{amsfonts}
\usepackage{float}
\usepackage{natbib}
\usepackage{blkarray}\usepackage{bbold}
\usepackage{ulem} %To strike out things using \sout
\usepackage{econometrics} % for bold greek letters, e.g. \valpha instead of \alpha
\usepackage{enumerate}
\usepackage{xcolor}  % Coloured text etc.
% 
\usepackage{multirow}
\usepackage{todonotes}
\newcommand{\afaire}[1]{\todo[linecolor=red,backgroundcolor=red!25,bordercolor=red]{#1}}
\usepackage{stmaryrd}
\usepackage{algorithm}
\usepackage{algorithmicx}
\usepackage{algpseudocode}

\usepackage{collcell}
\usepackage{colortbl,dcolumn}

\usepackage{dsfont}

\newcommand{\code}[1]{\colorbox{light-gray}{\texttt{#1}}}

\newcommand\thevector[4]{
\begin{tikzpicture}
\clip (-0.09,-0.13) rectangle + (.47,.32);
 \node [inner sep=0,outer sep=0,inner frame sep=0pt,tight background,draw=none] (first) at (0,0)  {$#1$};
\node [inner sep=0,outer sep=0,inner frame sep=0pt,tight background,draw=none,scale=.8] (second) at (0.135,0.05) {$#2$};
\node [inner sep=0,outer sep=0,inner frame sep=0pt,tight background,draw=none,scale=.64] (third) at (0.24,0.09) {$#3$};
\node [inner sep=0,outer sep=0,inner frame sep=0pt,tight background,draw=none,scale=.512] (fourth) at (0.33,0.125) {$#4$};  
\end{tikzpicture}
}
\newcommand\A{\thevector{\mathbf{1}}{0}{0}{0}}
\newcommand\C{\thevector{0}{\mathbf{1}}{0}{0}}
\newcommand\G{\thevector{0}{0}{\mathbf{1}}{0}}
\newcommand\T{\thevector{0}{0}{0}{\mathbf{1}}}

 
\def\cmd#1{\texttt{\color{red}\footnotesize $\backslash$#1}}
\def\env#1{\texttt{\color{blue}\footnotesize #1}}

% ------ CODE COLOR DEFINITION ------ %

\definecolor{codered}{rgb}{0.6,0,0}
\definecolor{codeblue}{rgb}{0,0,0.8}
\definecolor{codegreen}{rgb}{0,0.5,0}
\definecolor{almostwhite}{gray}{0.55}
\definecolor{codepurple}{rgb}{0.58,0,0.82}
\definecolor{backcolour}{rgb}{0.95,0.95,0.92}

\lstset{
    basicstyle=\ttfamily\small,
    keywordstyle=\bfseries\color{codeblue},
    emphstyle=\ttfamily\color{codered},   % Custom highlighting style
    stringstyle=\color{codepurple},
    numbers=left,
    numberstyle=\small\color{almostwhite},
    rulesepcolor=\color{red!20!green!20!blue!20},
    frame=shadowbox,
    commentstyle=\color{codegreen},
    captionpos=b    
}
\newcommand{\fr}[1]{#1}
\newcommand{\en}[1]{}


% ------------- PRESENTATION INFO --------------- %
\newcommand{\fullconference}{Discussion}
\newcommand{\shortconference}{Projects}
\newcommand{\contact}{}

\author[Daniel Bonnéry]{Daniel Bonnéry}
\institute[]{}
\title[Reconstitution de séquences de variants ]{Une méthode d’échantillonnage pour la reconstitution de séquences de variants de gènes d'antibiorésistance depuis des données métagénomiques}

\date[2024/09/30]{Séminaire, \small 3 juin 2024}
\begin{document}

% ------------ TITLE SLIDE --------------- %
{
% Remove headline and footline from first slide
\setbeamertemplate{footline}{} 
\setbeamertemplate{headline}{} 


\begin{frame}\label{start}
    \titlepage
        
    \begin{figure}
            \includegraphics[scale=0.15]{images/ensae.png} 
    \end{figure}
\end{frame}
}
\begin{frame}
        Daniel Bonnéry$^*$ (Ensae)\\
        Guillaume Kon Kam King (Inrae)\\
        Anne-Laure Abraham (Inrae)\\
	Ouleye Sidibe (Inrae)\\
	Sebastien Leclercq (Inrae)\\
        Nicolas Chopin (Crest-Ensae)
\end{frame}
\begin{frame}{Contents}
    \tableofcontents[sectionstyle=show, subsectionstyle=show/shaded/hide, subsubsectionstyle=show/shaded/hide]
\end{frame}

\section{\fr{Le modèle paramétrique et la question statistique}\en{The parametric model and the statistical question}}
\subsection{Observations}
\begin{frame}{Observations}

We observe a 3 dimensional array of counts.
$$\left(n_{v,s,a}\right)_{\footnotesize\begin{array}{c}
v\in\{1,\ldots,V\}\\
s\in\{1,\ldots,S\}\\
a\in \{1,\ldots,4\}\end{array}}$$

\begin{itemize}
\item $(n_{v,s,a})$ : \fr{comptages des nucléotides}\en{counts of nucleotides}
    \begin{itemize}
\item \fr{à la position}\en{at position} $v$
\item \fr{dans l'échantillon}\en{in sample} $s$
\item \fr{de type}\en{of type} $a$ :
    \begin{itemize}
    \item $a=1=A=(1,0,0,0)$
    \item $a=2=C=(0,1,0,0)$
    \item $a=3=G=(0,0,1,0)$
    \item $a=4=T=(0,0,0,1)$
    \end{itemize}
\end{itemize}
\end{itemize}
\end{frame}

\begin{frame}
    $$n_{.,s=1,.}=\begin{blockarray}{ccccc}
    &a=A&a=C&a=G&a=T&\\
    \begin{block}{c(cccc)}
 {\scriptscriptstyle v=1}&600&400&0&0\\   
 v=2&400&600&1&0\\
    \end{block}
\end{blockarray} $$


$$n_{.,s=2,.}=\begin{blockarray}{ccccc}
    &a=A&a=C&a=G&a=T&\\
    \begin{block}{c(cccc)}
{\scriptscriptstyle v=1}&200&800&0&1\\   
  {\scriptscriptstyle v=2}&800&200&0&0\\
    \end{block}
\end{blockarray} $$

\end{frame}

\subsection{Variants}

\begin{frame}
\en{It looks like we have two variants in these 2 samples:}
\fr{Il semblerait qu'il y a deux variants dans chacun des deux échantillons}

$$\tau=\begin{blockarray}{ccc}
    &&\\&\scriptscriptstyle g=1&\scriptscriptstyle g=2\\
    \begin{block}{c(cc)}
 {\scriptscriptstyle v=1}&A&C\\   
  {\scriptscriptstyle v=2}&C&A\\   
    \end{block}
\end{blockarray} =    \begin{blockarray}{ccc}
    &{\!\!\!\!a=}1^{2^{3^4}}&\\
    &\scriptscriptstyle g=1&\scriptscriptstyle g=2\\
    \begin{block}{c(cc)}
 {\scriptscriptstyle v=1}&\A&\C\\   
 {\scriptscriptstyle v=2}&\C&\A\\   
    \end{block}
\end{blockarray} $$

\begin{itemize}
    \item $\tau_{v,g,a}$ \fr{indique si}\en{indicates if}
        \begin{itemize}
            \item \en{at  position}\fr{à la position} $v$
            \item \en{the nucleotide of variant}\fr{le nucléotide du variant} $g$
            \item \fr{est}\en{is} $a$.
        \end{itemize}
\end{itemize}



\end{frame}

\subsection{Proportions}
\begin{frame}
        $$\pi=\begin{blockarray}{ccc}
    &\scriptscriptstyle s=1&\scriptscriptstyle s=2\\
    \begin{block}{c(cc)}
 {\scriptscriptstyle g=1}&0.6&0.2\\   
 {\scriptscriptstyle 
 g=2}&.04&0.8\\   
    \end{block}
\end{blockarray} $$
$\pi_{g,s}$ est la proportion de
 \begin{itemize}
    \item variant $g$
    \item dans l'échantillon $s$.
\end{itemize}
\end{frame}

\subsection{\en{Measurement errors}\fr{Erreur de mesure}}
\begin{frame}{\en{Measurement errors}\fr{Erreur de mesure}}
\en{The variable $\epsilon_{b,a}$ is the probability that the measurment of a nucleotide of type $b$ reads $a$}
\fr{La variable $\epsilon_{b,a}$ est la probabilité que la mesure d'un nucleotide de type $b$ donne  $a$}
  $$\epsilon=\begin{blockarray}{ccccc}
    &\scriptscriptstyle a=1&\scriptscriptstyle a=2&\scriptscriptstyle a=3&\scriptscriptstyle a=4\\
    \begin{block}{c(cccc)}
 {\scriptscriptstyle b=1}&\mathbf{0.91}&0.03&0.03&0.03\\   
 {\scriptscriptstyle 
 b=2}&0.03&\mathbf{0.91}&0.03&0.03\\   
  {\scriptscriptstyle 
 b=3}&0.02&0.02&\mathbf{0.94}&0.02\\   
  {\scriptscriptstyle 
 b=4}&0.05&0.04&0.01&\mathbf{0.90}\\   
    \end{block}
\end{blockarray} $$


\end{frame}
\subsection{\fr{Distribution du tableau de comptages}\en{Distribution of the counts}}
\begin{frame}{\fr{Distribution du tableau de comptages}\en{Distribution of the counts}}
    
\begin{eqnarray*}\lefteqn{\mathcal{L}\left(n | \pi, \tau,\epsilon \right)}\nonumber\\& =& \prod_{v=1}^{V} \prod_{s = 1}^{S} (n_{v,s,+})!\times\frac{\prod_{a = 1}^{4} \left( \sum_{g = 1}^{G}\sum_{b=1}^{4} \tau_{v,g,b}\epsilon_{b,a} \pi_{g,s}  \right)^{n_{v,s,a}}}{\prod_{a = 1}^{4}n_{v,s,a}!}\end{eqnarray*}
\end{frame}

\subsection{\fr{Approche bayesienne}}
\begin{frame}{\en{Approche bayesienne}}
\fr{Quelle est la loi de probabilité du résultat du lancé d'un dé sachant que le résultat est pair ?
{\bf A priori}, le dé est non pipé, et la loi $\eta$ du résultat est uniforme.
}

\end{frame}
\begin{frame}{\en{Approche bayesienne (2)}}
\fr{Règle de Bayes:}
$$\eta(\{x\}|\{2,4,6\})=\frac{\eta(\{x\}\cap\{2,4,6\})}{\eta(\{2\})+\eta(\{4\})+\eta(\{6\})}$$
\end{frame}
\begin{frame}{}
\fr{Notre problème est similaire. Une loi sur le vecteur de variables aléatoires }$(n,\pi,\tau,\epsilon)$
\fr{est construit à partir de lois a priori sur }$\tau$,$\pi$ \fr{et}\en{and} $\epsilon$, \fr{et à partir de la loi de $n$ sachant ces paramêtres.}

$$\eta(n,\pi,\tau,\epsilon)=  \eta(n\mid \tau,\pi,\epsilon)\times\eta(\tau)\times\eta(\pi)\times\eta(\epsilon)$$

La solution de notre problème est:

$$\eta(\pi,\tau,\epsilon\mid n)=\frac{\eta(n,\pi,\tau,\epsilon)}{\int \eta(n,\pi,\tau,\epsilon) \mathrm{d}\tau~\mathrm{d}\pi~\mathrm{d}\epsilon}$$


\end{frame}


\section{Desman}
\subsection{Présentation}
\begin{frame}


Quince, C., Delmont, T.~O., Raguideau, S., Alneberg, J., Darling, A.~E.,
  Collins, G., and Eren, A.~M. (2017).

Desman: a new tool for de novo extraction of strains from
  metagenomes.

{\em Genome biology}, 18:1--22.

\end{frame}
\begin{frame}
\begin{itemize}
\item Desman est un algorithme d'échantillonage qui permet d'obtenir un échantillon dont les propriétés sont proches d'un échantillon iid de la loi a posteriori de ($\tau$,$\pi$,$\epsilon$).
\item Il est basé sur des méthodes de Monte Carlo par chaînes de Markov
\item obtenues par échantillonnage de Gibbs
\item après introduction de variables latentes pour obtenir un maximum de lois conjuguées et éviter de devoir tirer avec Metropolis Hastings
\end{itemize}
\end{frame}
\subsection{Algorithme de Gibbs}
\begin{frame}{Algorithme de Gibbs}

    \begin{algorithm}[H]
        \caption{Algorithme de Gibbs}
        \begin{algorithmic}[1]
            \State Initialiser $\mathbf{\Theta}^{(0)} = (\Theta_1^{(0)}, \Theta_2^{(0)}, \ldots, \Theta_I^{(0)})$
            \For{$t = 1$ to $T$}
                \For{$i = 1$ to $I$}
                    \State Échantillonner 
                    \State \quad$\Theta_i^{(t)} \sim \eta\left(\Theta_i \mid \Theta_1^{(t)}, \ldots, \Theta_{i-1}^{(t)}, \Theta_{i+1}^{(t-1)}, \ldots, \Theta_n^{(t-1)}\right)$
                \EndFor
            \EndFor
        \end{algorithmic}
    \end{algorithm}
    
    
\en{The transition kernel is the distribution}\fr{On appelle noyau de transition la distribution } $P^{\Theta^{(t+1)}\mid\Theta^{(t)}}$ 
\end{frame}

\begin{frame}{Explications}
    \begin{itemize}
        \item L'algorithme commence par une initialisation de $\mathbf{\Theta}^{(0)}$.
        \item À chaque étape $t$, chaque variable $\Theta_i$ est mise à jour en échantillonnant de sa distribution conditionnelle.
        \item Le processus est répété pour un nombre d'itérations $T$.
        \item En fin de compte, les échantillons $\mathbf{\Theta}^{(T)}$ sont utilisés pour estimer la distribution cible.
    \end{itemize}
\end{frame}



\begin{frame}
$$\Theta_i^{(t)} \sim \eta\left(\Theta_i \mid \Theta_1^{(t)}, \ldots, \Theta_{i-1}^{(t)}, \Theta_{i+1}^{(t-1)}, \ldots, \Theta_n^{(t-1)}\right)$$
Comment faire ?

\begin{enumerate}
\item Lois conjuguées
\item Metropolis Hasting
\end{enumerate}
\end{frame}

\subsection{Les étapes de Gibbs dans Desman}
\begin{frame}
    
\begin{algorithm}[H]
\caption{\en{MCMC kernel}\fr{Noyau MCMC}}\label{alg:mcmck}
\begin{algorithmic}
\Procedure{$M$}{$n,\Theta=(\tau,\pi,\epsilon);\alpha_\pi,\alpha_\epsilon,e$}
    % \State Compute $\ell\left(n,\gamma^{(i-1)},\tau^{(i-1)},\epsilon^{(i-1)}\right)$\hfill\code{self.ll}
    % \State Compute $\ell p\left(n,\gamma^{(i-1)},\tau^{(i-1)},\epsilon^{(i-1)}\right)$\hfill\code{self.lp}
\State $(\nu,\mu)\gets{\mathrm{sample}_{(\nu,\mu)}(n,\Theta)}$
\State $\pi\gets\mathrm{sample}_\pi(\mu;\alpha_\pi)$
\State{$\tau\gets\mathrm{sample}_\tau(n,\tau,\pi,\epsilon)$}{} 
\State{$\epsilon\gets\mathrm{sample}_\epsilon(\nu;\alpha_\epsilon)$}{} 
\State Return $\left(\tau,\pi,\epsilon\right)$
\EndProcedure
\end{algorithmic}
\end{algorithm}

\end{frame}

\subsection{Identification d'un problème}
\begin{frame}
L'étape de Gibbs correspondante à $\tau$ semble problématique: $\tau$ ne varie pas avec $t$.
\end{frame}


\subsection{Diagnostics}


\begin{frame}
    $$n_{.,s=1,.}=\begin{blockarray}{ccccc}
    &a=A&a=C&a=G&a=T&\\
    \begin{block}{c(cccc)}
 {\scriptscriptstyle v=1}&600&400&0&0\\   
 v=2&400&600&1&0\\
    \end{block}
\end{blockarray} $$


$$n_{.,s=2,.}=\begin{blockarray}{ccccc}
    &a=A&a=C&a=G&a=T&\\
    \begin{block}{c(cccc)}
{\scriptscriptstyle v=1}&200&800&0&1\\   
  {\scriptscriptstyle v=2}&800&200&0&0\\
    \end{block}
\end{blockarray} $$

$$\overset{t=1}{\begin{blockarray}{ccc}
    &&\\&\scriptscriptstyle g=1&\scriptscriptstyle g=2\\
    \begin{block}{c(cc)}
 {\scriptscriptstyle v=1}&A&C\\   
  {\scriptscriptstyle v=2}&A&C\\   
    \end{block}
\end{blockarray}} \to \overset{t=2}{\begin{blockarray}{ccc}
    &&\\&\scriptscriptstyle g=1&\scriptscriptstyle g=2\\
    \begin{block}{c(cc)}
 {\scriptscriptstyle v=1}&A&C\\   
  {\scriptscriptstyle v=2}&C&C\\   
    \end{block}
\end{blockarray}} \to \overset{t=3}{\begin{blockarray}{ccc}
    &&\\&\scriptscriptstyle g=1&\scriptscriptstyle g=2\\
    \begin{block}{c(cc)}
 {\scriptscriptstyle v=1}&A&C\\   
  {\scriptscriptstyle v=2}&C&A\\   
    \end{block}
\end{blockarray}}$$
\end{frame}

\section{\en{Improvements}\fr{Stratégies}}






\subsection{Block sampling}

\begin{frame}
Il s'agit d'échantillonner en block $\tau_{v,.,.}$:


$$\overset{t=1}{\begin{blockarray}{ccc}
    &&\\&\scriptscriptstyle g=1&\scriptscriptstyle g=2\\
    \begin{block}{c(cc)}
 {\scriptscriptstyle v=1}&A&C\\   
  {\scriptscriptstyle v=2}&A&C\\   
    \end{block}
\end{blockarray}}  \to \overset{t=2}{\begin{blockarray}{ccc}
    &&\\&\scriptscriptstyle g=1&\scriptscriptstyle g=2\\
    \begin{block}{c(cc)}
 {\scriptscriptstyle v=1}&A&C\\   
  {\scriptscriptstyle v=2}&C&A\\   
    \end{block}
\end{blockarray}}$$

Problème:

\begin{table}[H]
\centering
\begin{tabular}{lr}
  \hline
$4^{1}$&$4$\\$4^{2}$&$16$\\$4^{3}$&$64$\\$4^{4}$&$256$\\$4^{5}$&$1~024$\\$4^{10}$&$1~048~576$\\$4^{15}$&$1~073~741~824$\\$4^{20}$&$1~099~511~627~776$\\   \hline
\end{tabular}
\end{table}

\end{frame}

\subsection{Fixed variants}

\begin{frame}
Il s'agit de remplacer la distribution a priori de $\tau$ par 
$\mathrm{Uniform}(\mathcal{T})$
\begin{itemize}
\item Metropolis Hasting est évité.
\item Satisfaisant lorsque les variants sont connus
\item Ne répond pas à l'objectif initial de détection.
\end{itemize}
\end{frame}
\subsection{Cibler $\rho$}

\begin{frame}


\begin{eqnarray*}\lefteqn{\mathcal{L}\left(n | \pi, \tau,\epsilon \right)}\nonumber\\& =& \prod_{v=1}^{V} \prod_{s = 1}^{S} (n_{v,s,+})!\times\frac{\prod_{a = 1}^{4} \left( \sum_{g = 1}^{G}\sum_{b=1}^{4} \tau_{v,g,b}\epsilon_{b,a} \pi_{g,s}  \right)^{n_{v,s,a}}}{\prod_{a = 1}^{4}n_{v,s,a}!}\nonumber\\
& =&\prod_{v=1}^{V} \prod_{s = 1}^{S} (n_{v,s,+})!\times\frac{\prod_{a = 1}^{4} \left( \sum_{g = 1}^{G} \rho_{v,g,a} \pi_{g,s}  \right)^{n_{v,s,a}}}{\prod_{a = 1}^{4}n_{v,s,a}!}\end{eqnarray*}

avec

$$\rho_{v,g,a}=\sum_{b=1}^4\tau_{vgb}\epsilon_{ba}$$
\end{frame}
\begin{frame}
 $$\rho_{.,g,.}=\begin{blockarray}{ccccc}
    &\scriptscriptstyle A&\scriptscriptstyle C&\scriptscriptstyle G&\scriptscriptstyle T\\
    \begin{block}{c(cccc)}
 {\scriptscriptstyle v=1}&0.03&\mathbf{0.91}&0.03&0.03\\   
 {\scriptscriptstyle 
 \vdots}&\vdots&\vdots&\vdots&\vdots\\   
  {\scriptscriptstyle 
 v=V}&0.05&0.04&\mathbf{0.90}&0.01\\   
    \end{block}
\end{blockarray} $$
\begin{itemize}
\item On utilise un hyperparamêtre de forme très petit pour la loi de Dirichlet, ce qui assure la concentration des $\rho_{v,g,.}$ autour des sommets A,C,G, ou T.
\item On garde des lois conjuguées pour $\rho$.
\item La mise à jour se fait naturellement par block.
\item Rajout d'un a priori non conjugué sur l'hyperparamêtre de forme.
\end{itemize}
\end{frame}


\begin{frame}[fragile]
\input{"4-simplex.tikz"}
\end{frame}




\begin{frame}
\begin{figure}[H]

     \begin{subfigure}[b]{0.49\textwidth}
\includegraphics[width=\textwidth,trim={3.7cm 0 7 0},clip]{simplex1.png}
\caption{$\alpha=0.1~\mathds{1}_3$}
\end{subfigure}%
     \begin{subfigure}[b]{0.49\textwidth}
\includegraphics[width=\textwidth,,trim={3.7cm 0 7 0},clip]{simplex2.png}
\caption{$\alpha=0.01~\mathds{1}_3$}
     \end{subfigure}
\caption{ \fr{Échantillon de distribution de Dirichlet sur un 3-simplexe} \en{Dirichlet sample on the 3-simplex}}
\end{figure}
\end{frame}

\subsection{SMC and Tempering}

\begin{frame}[fragile]
\includegraphics[scale=.55]{smc.png}
%\input{SMC.tikz}
\end{frame}


\begin{frame}[fragile]
\includegraphics[scale=.55]{tempering.png}
%\input{SMC.tikz}
\end{frame}

\begin{frame}
Stratégies de tempering:
\begin{itemize}
\item Puissance de la vraisemblance
\item Data tempering
\item Paramêtre de tempering naturel.
\end{itemize}
\end{frame}

\begin{frame}
Stratégie retenue
\begin{itemize}
\item Cibler $\tau$, $\epsilon$ et $\pi$
\item Relaxer $\tau$
\item Le paramêtre de forme sur $\tau$ devient le paramêtre de tempering naturel.
\end{itemize}
\end{frame}




\begin{frame}[plain]


\includegraphics[scale=.47]{models.png}
\end{frame}

\section{\en{Work in progress}\fr{Travaux en cours}}
\begin{frame}
\begin{itemize}
\item Développement d'algorithmes en R basés sur jags.
\item SMC
\item tempering.
\item \fr{Choix de modèle}\en{Model choice} ($G$)
\item ESS
\item Exploitation des trajectoires de particules obtenues.
\end{itemize}
\end{frame}




\begin{frame}[allowframebreaks]

\nocite{*}
\bibliographystyle{apalike} 

\bibliography{biblio} 

\end{frame} 



\end{document}